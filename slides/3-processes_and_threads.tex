\section{Processes and Thread}
  \begin{frame}
    \frametitle{Processes and Thread}
        \begin{block}{Overview}
          \begin{itemize}
            \item Why processes are so important?
            \item What differences between processes and threads?
          \end{itemize}
        \end{block}
  \end{frame}

  \begin{frame}
    \frametitle{Processes}
        \begin{itemize}
          \item Most important abstraction,
          \item Turn single-CPU into multiple virtual CPU
          \item Enable pseudo concurrent operations (pseudoparallelism),
          \item Without modern computing could not exist.
        \end{itemize}
  \end{frame}

  \begin{frame}
    \frametitle{Processes}
        \begin{block}{Processes}
          \begin{itemize}
            \item are instance of executing program
            \item include currents values of
              \begin{itemize}
                \item program counter,
                \item registers,
                \item variables.
              \end{itemize}
            \item have their own virtual CPU (multiprogramming) -- considerations about time management, RTS.
          \end{itemize}
        \end{block}
  \end{frame}

  \begin{frame}
    \frametitle{The student partying (a fictional analogy)}
        \begin{itemize}
          \item The student, at a home party, makes a cocktail.
          \begin{itemize}
            \item Student: CPU, recipe: program, drinks: data, glasses: resource, action: process.
          \end{itemize}
          \item While pouring the last ingredient her/his phone rang and s/he answers.
          \begin{itemize}
            \item Student: CPU, phone-skill: program, phone: resource, phone call details: data, action: process.
          \end{itemize}
        \end{itemize}
	A process is an activity having a program, input(s), output(s) and a state. OS uses scheduling algorithm to determines when to stop/start which process.
  \end{frame}

  \begin{frame}
    \frametitle{Processes}
        \begin{block}{Creation}
          \begin{itemize}
            \item System initialization,
            \item Process creation done by a running process,
            \item User request,
            \item Initiation of a batch jobs.
          \end{itemize}
        \end{block}

        \begin{block}{Termination}
          \begin{columns}[onlytextwidth]
            \begin{column}{0.4\textwidth}
              \begin{itemize}
                \item Voluntary
                \begin{itemize}
                  \item Normal exit,
                  \item Error exit.
                \end{itemize}
              \end{itemize}
            \end{column}
            \begin{column}{0.6\textwidth}
              \begin{itemize}
                \item Involuntary
                \begin{itemize}
                  \item Fatal error,
                  \item Killed by another process.
                \end{itemize}
              \end{itemize}
            \end{column}
          \end{columns}
        \end{block}
  \end{frame}

  \begin{frame}
    \frametitle{Process states}
        \begin{block}{State}
          \begin{enumerate}[a.]
            \item Running,
            \item Ready,
            \item Blocked.
          \end{enumerate}
        \end{block}
        \begin{block}{Transition}
          \begin{enumerate}[1.]
            \item Scheduler pick another process.
            \item Scheduler pick this process.
            \item Input available.
            \item Input required.
          \end{enumerate}
        \end{block}
  \end{frame}

  \begin{frame}
    \frametitle{Threads}
        \begin{block}{Threads}
          \begin{enumerate}
            \item Processes within a process.
            \item Enable to decompose big task into multiple sequential smaller tasks ..
            \item .. while \textbf{sharing} a memory space.
            \item Easier, faster, to create and destroy than processes as they are lighter.
          \end{enumerate}
        \end{block}
	Web browser example with multiple threads.
  \end{frame}

  \begin{frame}
    \frametitle{Threads}
        \begin{block}{Concurrent programming: race condition (example)}
          \begin{itemize}
            \item Thread S, and N, respectively correspond to the South, and North, gate in a Park.
            \item They count the current number of visitors:
            \begin{itemize}
              \item When a visitor enters, the counter is incremented by one.
              \item When a visitor leaves, the counter is decremented by one.
            \end{itemize}
            \item Both threads share a common variable, the counter.
            \item They run on the same computer and, thus, share a sole CPU.
            \item To increment a variable at least three operations are required:
            \begin{enumerate}
              \item Read the variable.
              \item Compute the incrementation.
              \item Write the new value of the variable.
            \end{enumerate}
          \end{itemize}
        \end{block}
	What could go wrong?
  \end{frame}

  \begin{frame}
    \frametitle{Q/A}
    \begin{itemize}
        \item Why interruption handler are usually written in ASM? % Higher-level languages do not allow the hardware access needed.
        \item Process states: can a process go directly from [Blocked] to [Running], or from [Ready] to [Blocked]? % [Blocked] -> [Running] if I/O is done and CPU is idle, [Ready] -> [Blocked] is not possible
        \item Why would a thread give up its CPU time, call a thread\_yield? % Thread cooperate, they usually are written by the same programmer. If a thread need to free some CPU time for other thread, it's ok.
        \item What is an advantage and a inconvenient to implement thread in user space? % no need to trap but if a thread is blocked it stays blocked.
    \end{itemize}
  \end{frame}
