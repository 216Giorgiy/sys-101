\section{File System}

\begin{frame}
  \frametitle{File System}
  \begin{block}{Challenges}
    \begin{itemize}
      \item How to store very large amount of data?
      \item How to store it after {\emph the} process has been killed?
      \item How to allow multiple processes to access it concurrently?
    \end{itemize}
  \end{block}
\end{frame}

\begin{frame}
  \frametitle{File System}
  \begin{block}{Two operations}
    \begin{itemize}
      \item read\_block(block\_number)
      \item write\_block(block\_number)
    \end{itemize}
  \end{block} \pause
  \begin{block}{How to, with these two, ..}
    \begin{itemize}
      \item Find information?
      \item Keep a user from reading another user's data?
      \item Know which blocks are free?
    \end{itemize}
  \end{block}
\end{frame}

\begin{frame}
  \frametitle{File Abstraction}
  OSes abstract processor into process, physical memory into virtual address space and file system into files. \\
  Processes (Threads), address space and files are the three most important abstractions that an Operation System offers.
\end{frame}

\begin{frame}
  \frametitle{File Naming}
  \begin{block}{Naming rules}
    \begin{itemize}
      \item Different rules for different OS.
      \item Names length varies from 1 to 255 characters.
      \item Case-sensitive (UNIX), or not (MS-DOS).
      \item Extension consideration.
    \end{itemize}
  \end{block}
\end{frame}

\begin{frame}
  \frametitle{Files Structure}
  \begin{block}{File Structure}
    \begin{enumerate}
      \item Unstructured sequence of byte (any meaning is imposed by the user procces). Offers the maximum flexibility.\footnote{UNIX, MS-DOS, Windows}
      \item Fixed length-records each with internal structure (80 characters records, 132 char for line printer).\footnote{Not used anymore}
      \item Tree records, each files has a key value, tree sorted according to these values.\footnote{Still used in some commercial data processing}
    \end{enumerate}
  \end{block}
\end{frame}

\begin{frame}
  \frametitle{Files Types}
  \begin{block}{File Types}
    \begin{itemize}
      \item Regular files (ASCII or binary files).
      \item Names length varies from 1 to 255 characters.
      \item Case-sensitive (UNIX), or not (MS-DOS).
      \item Extension consideration.
    \end{itemize}
  \end{block}
\end{frame}

\begin{frame}
  \frametitle{Binary File Example}
  \begin{block}{Executable binary example:}
    \begin{itemize}
      \item header,
      \begin{itemize}
          \item Magic-number (identifying the file as executable),
          \item Size of the various pieces of the file,
          \item Address at which execution starts
          \item Various flags.
      \end{itemize}
      \item text,
      \item data,
      \item relocation bits,
      \item symbol table (for debugging purposes).
    \end{itemize}
  \end{block}
\end{frame}

\begin{frame}
  \frametitle{Binary File Example}
  \begin{block}{Archive example (collection of library procedures compiled but not linked):}
    \begin{itemize}
      \item header (telling its name),
      \item Creation date,
      \item Owner,
      \item Protection code,
      \item Size.
    \end{itemize}
  \end{block}
\end{frame}

\begin{frame}
  \frametitle{Files Extension and OS}
  \begin{block}{OS details}
    \begin{itemize}
      \item An OS must at least recognize its own executable file type (they usually recognized more than this one type),
      \item TOPS-20 checked the creation date of executable file, sought for its source file, and recompile it if the source file was updated.
      \item \emph{make} program reproduce this behavior, file extensions are mandatory.
      \item Strong mandatory file extensions make an OS unsuable (e.g., all extension file output produced are ".dat" -- impossible to copy a file!)
    \end{itemize}
  \end{block}
\end{frame}

\begin{frame}
  \frametitle{Files Access}
  \begin{block}{OS details}
    \begin{itemize}
      \item An
    \end{itemize}
  \end{block}
\end{frame}
