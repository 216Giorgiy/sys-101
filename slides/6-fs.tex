\section{File System}

\begin{frame}
  \frametitle{File System}
  \begin{block}{Challenges}
    \begin{itemize}
      \item How to store very large amount of data?
      \item How to store data after {\emph the} process has been killed?
      \item How to allow multiple processes to access it concurrently?
    \end{itemize}
  \end{block}
\end{frame}

\begin{frame}
  \frametitle{File System}
  \begin{block}{Two operations}
    \begin{itemize}
      \item read\_block(block\_number)
      \item write\_block(block\_number)
    \end{itemize}
  \end{block} \pause
  \begin{block}{How to, with these two, ..}
    \begin{itemize}
      \item Find information?
      \item Keep a user from reading another user's data?
      \item Know which blocks are free?
    \end{itemize}
  \end{block}
\end{frame}

\begin{frame}
  \frametitle{File Abstraction}
  OSes abstract processor into process, physical memory into virtual address space and file system into files. \\
  Processes (Threads), address space and files are the three most important abstractions that an Operation System offers.
\end{frame}

\begin{frame}
  \frametitle{File Naming}
  \begin{block}{Naming rules}
    \begin{itemize}
      \item Different rules for different OS.
      \item Names length varies from 1 to 255 characters.
      \item Case-sensitive (UNIX), or not (MS-DOS).
      \item Extension consideration.
    \end{itemize}
  \end{block}
\end{frame}

\begin{frame}
  \frametitle{Files Structure}
  \begin{block}{File Structure}
    \begin{enumerate}
      \item Unstructured sequence of byte (any meaning is imposed by the user procces). Offers the maximum flexibility.\footnote{UNIX, MS-DOS, Windows}
      \item Fixed length-records each with internal structure (80 characters records, 132 char for line printer).\footnote{Not used anymore}
      \item Tree records, each files has a key value, tree sorted according to these values.\footnote{Still used in some commercial data processing}
    \end{enumerate}
  \end{block}
\end{frame}

\begin{frame}
  \frametitle{Files Types}
  \begin{block}{Regular Files}
    \begin{itemize}
      \item ASCII
      \item Binary files
    \end{itemize}
  \end{block}
\end{frame}

\begin{frame}
  \frametitle{Binary File Example}
  \begin{block}{Executable binary example:}
    \begin{itemize}
      \item header,
      \begin{itemize}
          \item Magic-number (identifying the file as executable),
          \item Size of the various pieces of the file,
          \item Address at which execution starts
          \item Various flags.
      \end{itemize}
      \item text,
      \item data,
      \item relocation bits,
      \item symbol table (for debugging purposes).
    \end{itemize}
  \end{block}
\end{frame}

\begin{frame}
  \frametitle{Binary File Example}
  \begin{block}{Archive example (collection of library procedures compiled but not linked):}
    \begin{itemize}
      \item header (telling its name),
      \item Creation date,
      \item Owner,
      \item Protection code,
      \item Size.
    \end{itemize}
  \end{block}
\end{frame}

\begin{frame}
  \frametitle{Files Extension and OS}
    \begin{itemize}
      \item An OS must at least recognize its own executable file type (they usually recognized more than this one type),
      \item TOPS-20 checked the creation date of executable file, sought for its source file, and recompile it if the source file was updated.
      \item \emph{make} program reproduce this behavior, file extensions are mandatory.
      \item Strong mandatory file extensions make an OS unusable (e.g., all extension file output produced are ".dat" -- impossible to copy a file!)
    \end{itemize}
\end{frame}

\begin{frame}
  \frametitle{File Access}
    \begin{itemize}
      \item Sequential access: files were read from the beginning to the end (magnetic tape).
      \item Random access files: files can be read in any order. To choose the position in the file and read you can:
      \begin{itemize}
          \item Give the position as a parameter at each read operation,
          \item Seek at a specific position and then sequentially read (UNIX and Windows).
      \end{itemize}
    \end{itemize}
\end{frame}

\begin{frame}
    \frametitle{File Attributes (Metadata)}
    \begin{itemize}
      \item Rights: who and how can the file be accessed.
      \item Password: required to access the file.
      \item Creator: ID of the user who created the file.
      \item Owner: ID of the user who owns the file.
      \item Read-only flag.
      \item Hidden flag.
      \item System flag.
      \item Archive flag (set by the system when changed, cleared by the back-up program).
    \end{itemize}
\end{frame}

\begin{frame}
    \frametitle{File Attributes (Metadata)}
    \begin{itemize}
      \item ASCII/binary flag: 0 for ASCII file, 1 for binary file.
      \item Random access flag: 0 for sequential access only, 1 for random access.
      \item Temporary flag: 0 for normal, 1 for delete file on process exit
      \item Lock flag: 0 for unlocked, nonzero for locked.
      \item Creation time.
      \item Time of last access.
      \item Time of last change.
      \item Current size.
\end{itemize}
\end{frame}

\begin{frame}
  \frametitle{File Operations}
    \begin{itemize}
      \item Create: while no data are put into, the file is announced and some of its attributes are set.
      \item Delete: frees up some disk space (sys-call).
      \item Open: the system fetchs attributes and list of disk addresses into main memory for rapid access later on.
      \item Close: frees up internal table space. Some OS impose a maximum number of open files.
      \item Read: the user must specify how many data are expected and a buffer to put them in.
      \item Write: data are written. If position is the end of the file, the file size increases, if it is in the middle, existing data are overwritten.
      \item Append: write-like but only at the end of the file.
    \end{itemize}
\end{frame}

\begin{frame}
  \frametitle{File Operations}
    \begin{itemize}
      \item Seek: for random access files, set the file pointer at the requested position.
      \item Get attributes: many programs need it, i.e.\emph{make}.
      \item Set attributes: allows the user to modify some of the file attributes.
      \item Rename: simple as that.
    \end{itemize}
    \color{blue}\href{https://github.com/p-e-w/maybe}{GH.com/p-e-w/maybe}
\end{frame}

\begin{frame}
  \frametitle{Directories}
    \begin{itemize}
      \item Single-level directory system: only \textbf{one} folder (digital cameras, music players, first PCs).
      \item Hierarchical directory system.
    \end{itemize}
\end{frame}

\begin{frame}
  \frametitle{Path Names}
    \begin{itemize}
      \item Absolute (from \textbf{root} directory): \\ \emph{/etc/apt/source.list}, \emph{/var/log/messages}.
      \item Relative (from \textbf{current} directory):
        \begin{itemize}
          \item Implicit: \emph{Download/Gramatik.zip}, \emph{../Movies/}.
          \item Explicit: \emph{./Download/Gramatik.zip}, \emph{./../Movies/}.
        \end{itemize}
    \end{itemize}
    What are dot and dotdot?
\end{frame}

\begin{frame}
  \frametitle{Directory Operation}
    \begin{itemize}
      \item Create: creates a directory with dot and dotdot.
      \item Delete: deletes a directory (only empty directories can be deleted).
      \item Opendir: opens a directory so it can read.
      \item Readdir: reading a directory allows to list all files within it.
      \item Closedir: frees up internal table space.
      \item Rename: simple as that.
      \item Link: (hard link) increments the file's i-node counter and allows to make the file appear in different directories.
      \item Unlink: decrements the file's i-node counter. If it is zero the file is delete from the system, otherwise only the path name specified is removed (UNIX system call to delete a file is unlink).
    \end{itemize}
\end{frame}

\begin{frame}
  \frametitle{File System Layout}
    \begin{itemize}
      \item Sector 0 contains the MBR\footnote{Master Boot Record}, used to boot the computer, at the end of the MBR is the partition table.
      \item Partition table contains starting and ending disk addresses of the different partitions. One is marked as active.
      \item At boot time, the BIOS reads and executes the MBR, the MBR program locates the active partition, reads its first block (boot block) and execute it.
      \item The boot block loads the OS contained in that partition.
        \begin{itemize}
          \item All partitions contain a boot block, even if it does not contain a bootable OS.
        \end{itemize}
  \end{itemize}
\end{frame}

\begin{frame}
  \frametitle{Partition System Layout}
    \begin{itemize}
      \item Boot block: cf previous slide.
      \item Superblock: details include a magic number to identify the file system type and the number of blocks in the file system.
      \item Free space management: can be in a form of a list of pointers.
      \item I-nodes: array of data structure, one per file.
      \item Root directory: contains the top of the system tree.
      \item Files and directories: remainder of the disk.
  \end{itemize}
\end{frame}

\begin{frame}
  \frametitle{Implementing Files}
  How to keep track of which disk blocks match with which file?
    \begin{itemize}
      \item Contiguous allocation,
      \item Linked list allocation,
      \item Linked list allocation using a table in memory,
      \item I-nodes\footnote{Index-nodes}.
  \end{itemize}
\end{frame}

\begin{frame}
    \frametitle{Implementing Files: contiguous allocation}
    Each file occupies consecutive disk blocks.
    \begin{block}{Advantages}
        \begin{itemize}
            \item Simple to implement:
                \begin{itemize}
                    \item Keeping track of a file's block requires only two numbers: starting disk address, number of blocks.
                \end{itemize}
            \item Read performance is excellent (the entire file can be read from the disk in a single operation).
        \end{itemize}
    \end{block}
    \begin{block}{Drawbacks}
        \begin{itemize}
            \item Fragmentation.
            \item Disk space wasted.
            \item What about a file becoming bigger?
        \end{itemize}
    \end{block}
\end{frame}

\begin{frame}
    \frametitle{Implementing Files: linked list allocation}
    Each file is matched with a linked list of disk blocks.
    \begin{block}{Advantages}
        \begin{itemize}
            \item Every block of the disk can be used (no waste, no fragmentation).
            \item Directories entries only need to store the disk address of the first block (no matter how big are the files).
        \end{itemize}
    \end{block}
    \begin{block}{Drawbacks}
        \begin{itemize}
            \item Performance
                \begin{itemize}
                    \item Many head rotation may be required.
                    \item Amount of data storage in a block is not a power of two.%, but many programs read/write blocks whose size is a power of two.
                \end{itemize}
            \item Each block starts with the address of the next block%, that generates extra overhead.
        \end{itemize}
    \end{block}
\end{frame}

\begin{frame}
    \frametitle{Implementing Files: linked list allocation using a table in memory}
    Let's put the pointers in a table in memory (FAT).
    \begin{block}{Advantages}
        \begin{itemize}
            \item Every block of the disk can be used (no waste, no fragmentation).
            \item Random access is much easier as no disk references are needed (as these references are in the memory).
        \end{itemize}
    \end{block}
    \begin{block}{Drawbacks}
        \begin{itemize}
            \item Memory usage. The whole table must be in memory all the time.
                \begin{itemize}
                    \item 200-GB disk, 1-kB block size, entries of 3 bytes: 600 MB.
                \end{itemize}
            %\item Does not scale well.
        \end{itemize}
    \end{block}
\end{frame}

\begin{frame}
    \frametitle{Implementing Files: i-nodes}
    The index lists the attributes and disk addresses of the file's blocks.
    \begin{block}{Advantages}
        \begin{itemize}
            \item The memory is occupied only when file is open.
            \item The footprint is smaller, the RAM usage does not depends on the disk size (but the number of opened files).
        \end{itemize}
    \end{block}
    Indexes have the same size, thus, for big files, the last disk address point for an address of a block containing more disk block addresses.
\end{frame}

\begin{frame}
    \frametitle{Implementing Directories}
    Main function of directories: map the ASCII name of the file onto the information needed to locate the data.
    \begin{itemize}
        \item Directories entries provide the needed details to find file's disk blocks
            \begin{itemize}
                \item Disk address of the entire file,
                \item Number of the first block,
                \item Number of the i-node.
            \end{itemize}
    \end{itemize}
\end{frame}

\begin{frame}
    \frametitle{Implementing Directories: File Names' Length}
    How to deal with file names' length?
    \begin{itemize}
        \item Fixed-length
        \begin{itemize}
            \item MS-DOS: 1-8 character base name and optional extension of 1-3 characters.
            \item UNIX v7: 1-16 characters, including any extensions.
        \end{itemize}
        \item Variable-length
        \begin{itemize}
            \item Set a maximum limit and use it for the design.
            \item In-line.
            \item In a heap.
        \end{itemize}
    \end{itemize}
\end{frame}

%\begin{frame}
%    \frametitle{Shared Files}
%    \begin{itemize}
%        \item Disk blocks are listed in a data structure\footnote{the i-node in Linux} associated with the shared file, not in directories.
%        \item Symbolic linking.
%    \end{itemize}
%\end{frame}

\begin{frame}
    \frametitle{Other File Systems}
    \begin{itemize}
        \item Log-Structured File System (LFS).
        \begin{itemize}
          \item Computer speed bottleneck is the hard drive speed. reads are OK, writes are buffered and done by burst.
        \end{itemize}
        \item Journaling File System (NTFS, ext3).
        \begin{itemize}
          \item The OS keep a log of what the FS is going to do before it does it, so if a crash occurs, on the reboot it can repair it.
        \end{itemize}
    \end{itemize}
\end{frame}

\begin{frame}
    \frametitle{Virtual File Systems}
    A VFS integrates multiple file systems into an orderly structure.
    \begin{itemize}
        \item /: ext4 on an SSD.
        \item /media/cd-rom: ISO 9660.
        \item /home: NFS.
        \item /usr: NTFS on a HDD.
    \end{itemize}
    While there are 4 different FS, users access any file with the standard POSIX calls.
\end{frame}

\begin{frame}
    \frametitle{Defragmentation}
    Free blocks become scattered all over the disk among the used blocks. Big files are then stored in many places that slows read-calls.
\end{frame}

\begin{frame}
    \frametitle{Q/A}
    \begin{enumerate}
      \item Why a magic number is given for executable files whereas random ones are given to other files (UNIX - old versions)?
            % The magic number is the `BRANCH` instruction that is the first word of the file. Thus the file can be charged in memory and executed without knowing the size of the header!
      \item Does the rename() system call differ with the copy-then-remove operations?
            % Yes: the creation date is changed, if the disk is full it will fail, and it is slower to perform.
    \end{enumerate}
\end{frame}
