\section{OS Concepts}
  \begin{frame}
    \frametitle{OS Concepts}
        \begin{block}{Overview}
          \begin{itemize}
            \item Processes,
            \item Address spaces,
            \item Files,
            \item Input/Output,
            \item Permissions.
          \end{itemize}
        \end{block}
  \end{frame}

  \begin{frame}
    \frametitle{Processes}
    A process is a program being executed.
    \begin{block}{Each process:}
      \begin{itemize}
        \item has an address space (\textbf{core image}),
        \item has a register (program counter and stack pointer),
        \item has a list of open files,
        \item has a list of related processes,
        \item and all the details needed to run a program.
      \end{itemize}
    \end{block}
  \end{frame}

  \begin{frame}
    \frametitle{Process}
      \begin{block}{OS management of Processes}
        \begin{itemize}
          \item Execute,
          \item Save execution state (file pointers list, number of bytes to be read next) in a \textbf{process table}\footnote{Array or linked list},
          \item Stop.
        \end{itemize}
      \end{block}
    A process corresponds to its \textbf{core image} and its \textbf{process table entry}.
  \end{frame}
  \begin{frame}
    \frametitle{Child Process}
      \begin{block}{Process life}
        \begin{itemize}
          \item A \textbf{system call} starts a process.
          \item Binary code is executed.
            \begin{itemize}
              \item The process can create other processes, called \textbf{child processes} (and so on -- tree).
              \item Processes can communicate together using \textbf{IPC} means\footnote{Inter Process Communication}.
            \end{itemize}
          \item The OS may send \textbf{alarm} signal (interruption) to the process.
          \item The process executes a \textbf{system call} to terminates itself.
        \end{itemize}
      \end{block}
  \end{frame}
  \begin{frame}
    \frametitle{Users Process}
      \begin{block}{Process life}
        \begin{itemize}
          \item \textbf{UID}\footnote{User Identification} is a unique number assigned to each system user.
          \item Every process started has the UID of the user who started it.
          \item Every child process has the UID of its parent.
          \item One UID is called the super-user. The super-user has all permission.
          \item Users may also be members of groups. Each group has a \textbf{GID}.
        \end{itemize}
      \end{block}
  \end{frame}

  \begin{frame}
    \frametitle{Address Space}
    An address space is a memory location (from 0 to some maximum) and contains:
      \begin{itemize}
        \item executable program,
        \item program's data,
        \item program's stack.
      \end{itemize}
  \end{frame}
  \begin{frame}
    \frametitle{Address Space}
      \begin{itemize}
        \item Physical memory,
        \item Virtual memory (swap).
      \end{itemize}
  \end{frame}


  \begin{frame}
    \frametitle{File}
    OS hide all peculiar disk operations to offer abstracted model of device-independent file management.
      \begin{itemize}
        \item System calls are required to:
        \begin{itemize}
          \item Create, remove, read and write files; create and remove directories.
        \end{itemize}
        \item File system also match a tree structure.
      \end{itemize}
  \end{frame}
  \begin{frame}
    \frametitle{File}
      \begin{itemize}
        \item Path
        \item Several files may have the same name.
        \item Each file has a unique absolute path (and an infinity of relative ones).
        \item Mounted file system, merging trees.
      \end{itemize}
  \end{frame}
  \begin{frame}
    \frametitle{Special Files}
      I/O devices are abstracted to be used through same system calls as files do.
      \begin{itemize}
        \item Devices:
          \begin{itemize}
            \item Block files,
            \item Character files.
            \item Special files are kept in /dev.
          \end{itemize}
        \item Pipe:
          \begin{itemize}
            \item IPC mean,
            \item a special system call needs to be performed to known it's not a real file.
          \end{itemize}
      \end{itemize}
  \end{frame}

  \begin{frame}
    \frametitle{Tree}
      \begin{itemize}
        \item Both process and file are structured as tree.
        \item Process tree are usually not very deep, unlike file trees.
        \item Process hierarchy are usually short-lived (minutes or less) while directories may exist for years.
        \item Ownership and protection differs too.
      \end{itemize}
  \end{frame}

  \begin{frame}
    \frametitle{Permissions}
    \center{\textbf{\Large{
	{\color{OliveGreen}u}
	{\color{Marroon}g}
	{\color{fuchsia}o}
        }
        :
	{\color{OliveGreen}user}
	{\color{Marroon}group}
	{\color{fuchsia}other}
      }}
      \begin{itemize}
        \item Three 3-bit fields
          \begin{itemize}
            \item r{\color{gray}ead}
            \item w{\color{gray}rite}
            \item {\color{gray}e}x{\color{gray}ecute}
          \end{itemize}
        \item {\color{OliveGreen}rwx} {\color{Marroon}rwx} {\color{fuchsia}rwx} do-what-ever-you-want-file
        \item {\color{OliveGreen}rwx} {\color{Marroon}rwx} {\color{fuchsia}r-x} web-file
        \item {\color{OliveGreen}rw-} {\color{Marroon}r-x} {\color{fuchsia}r-x} virus
        \item {\color{OliveGreen}r-x} {\color{Marroon}---} {\color{fuchsia}---} personal-backup.tgz
      \end{itemize}
    \begin{figure}
      \centering
      \begin{tabular}{l|c|c}
        Right  & File  & Directory \\ \hline
        r & can read   & can list files \\ \hline
        w & can write  & can add/delete files \\ \hline
        x & can execute& can go through \\ \hline
      \end{tabular}
      \caption{Permissions meaning}
      \label{fig:tab-right}
    \end{figure}
  \end{frame}


  \begin{frame}
    \frametitle{Boot procedure}
    The BIOS\footnote{Basic Input Output System: low level I/O software system program present on the parentboard} contains the procedures to read the keyboard, write to the screen, perform I/O on disk. Held in RAM, OS can modify it when bugs are found.
    \begin{enumerate}
      \item The BIOS is started,
      \item The BIOS checks how much RAM is available,
      \item The BIOS verifies if keyboard, mouse (and other basic devices) are correctly installed and responding,
      \item The BIOS scans PCI\footnote{Peripheral Component Interconnect} and ISA\footnote{Industry Standard Architecture} buses to detect attached devices,
        \begin{enumerate}
          \item If new devices are found since last boot, these new devices are configured.
        \end{enumerate}
    \end{enumerate}
  \end{frame}

  \begin{frame}
    \frametitle{Boot procedure}
    \begin{enumerate}
    \setcounter{enumi}{4}
      \item The BIOS determines the boot device (floppy? CD-ROM? DVD? USB? ...?) by using a list of devices stored in CMOS memory,
      \item The program contained at the first sector of the boot device is executed,
        \begin{enumerate}
          \item It usually examines the partition table at the end of the boot sector to determine which partition is active.
          \item A secondary boot loader is read from that partition.
          \item This loader reads the operation system from the active partition and starts it.
        \end{enumerate}
      \item The OS queries the BIOS to get configuration details,
      \item The OS checks the driver of each device,
      \item The OS loads all these modules into the kernel,
      \item The OS initializes its table, creates background processes and starts up a login program.
    \end{enumerate}
  \end{frame}

  \begin{frame}
    \frametitle{Q/A}
    \begin{itemize}
      \item Why is the use of a process table in a timesharing system? Is it needed in single-process systems? % This table is needed to save the state of a suspended process. It's useless for single-process systems as the processes are not suspended
      \item Why do process trees not usually last for years? % It would require the system not to be rebooted
      \item In which situation a file tree would last for a few minutes? % Temporary files
      \item How to recognize a virus permission? % read the slides, and think a little bit
      \item What is the condition for N files to have the same name? % be in N different directories
      \item What is the condition for N files to have the same path? % To be the same file
      \item Why does a mounting point should be empty? % otherwise stored files are not accessible anymore
      \item Why a system would need \emph{virtual} memory? % because physical memory cost a lot while requiring more memory
    \end{itemize}
  \end{frame}
