\section{OS Concepts}
  \begin{frame}
    \frametitle{OS Concepts}
        \begin{block}{Overview}
          \begin{itemize}
            \item Processes,
            \item Address spaces,
            \item Files,
            \item Input/Output,
            \item Permissions,
            \item Shell.
          \end{itemize}
        \end{block}
  \end{frame}

  \begin{frame}
    \frametitle{Processes}
    A process is a program being executed. Each process:
    \begin{itemize}
      \item has an address space (\textbf{core image}),
      \item has a register (program counter and stack pointer),
      \item has a list of open files,
      \item has a list of related processes
      \item and all the details needed to run a program.
    \end{itemize}
  \end{frame}
  \begin{frame}
    \frametitle{Process}
      \begin{block}{OS management of Processes}
        \begin{itemize}
          \item Execute,
          \item Save execution state (file pointers list, number of bytes to be read next) in a \textbf{process table}\footnote{An array or linked list},
          \item Stop.
        \end{itemize}
      \end{block}
    A process corresponds to its \textbf{core image} and its \textbf{process table entry}.
  \end{frame}
  \begin{frame}
    \frametitle{Child Process}
      \begin{block}{Process life}
        \begin{itemize}
          \item A \textbf{system call} starts a process.
          \item Binary code is executed.
            \begin{itemize}
              \item The process can create other processes, called \textbf{child processes} (and so on -- tree).
              \item Processes can communicate together using \textbf{IPC}\footnote{Inter Process Communication}.
            \end{itemize}
          \item The OS may send \textbf{alarm} signal (interruption) to the process.
          \item The process executes a \textbf{system call} to terminates itself.
        \end{itemize}
      \end{block}
  \end{frame}
  \begin{frame}
    \frametitle{Users Process}
      \begin{block}{Process life}
        \begin{itemize}
          \item \textbf{UID}\footnote{User Identification} is a unique number assigned to each system user.
          \item Every process started has the UID of the user who started it.
          \item Every child process has the UID of its parent.
          \item One UID is called the super-user. The super-user has all permission.
          \item Users may also be members of groups. Each group has a \textbf{GID}.
        \end{itemize}
      \end{block}
  \end{frame}

  \begin{frame}
    \frametitle{Address Space}
    An address space is a memory location (from 0 to some maximum) and contains:
      \begin{itemize}
        \item executable program,
        \item program's data,
        \item its stack.
      \end{itemize}
  \end{frame}
  \begin{frame}
    \frametitle{Address Space}
      \begin{itemize}
        \item Physical memory,
        \item Virtual memory.
      \end{itemize}
  \end{frame}


  \begin{frame}
    \frametitle{File}
    OS are expected to hide all peculiar disk operations to offer abstracted model of device-independent file management.
      \begin{itemize}
        \item System calls are required to:
        \begin{itemize}
          \item Create, remove, read and write files; create and remove directories.
        \end{itemize}
        \item File system also match a tree structure.
      \end{itemize}
  \end{frame}
  \begin{frame}
    \frametitle{File}
      \begin{itemize}
        \item Path
          \begin{itemize}
            \item Absolute: from the root directory, starting with /
            \item Relative: from the current directory\footnote{each process has a current working directory. A system call allows process to change their working directory}, starting with "./" or a directory name.
          \end{itemize}
        \item Several files may have the same name.
        \item Each file has a unique absolute path (and an infinity of relative ones).
        \item Mounted file system, merging trees.
        \item Special files (so I/O devices are abstracted to be used through same system calls as files do):
          \begin{itemize}
            \item Block files,
            \item Character files.
            \item Special files are kept in /dev.
          \end{itemize}
      \end{itemize}
  \end{frame}
  \begin{frame}
    \frametitle{Special Files}
      I/O devices are abstracted to be used through same system calls as files do.
      \begin{itemize}
        \item Devices:
          \begin{itemize}
            \item Block files,
            \item Character files.
            \item Special files are kept in /dev.
          \end{itemize}
        \item Pipe:
          \begin{itemize}
            \item IPC mean,
            \item a special system call needs to be performed to known it's not a real file.
          \end{itemize}
      \end{itemize}
  \end{frame}

  \begin{frame}
    \frametitle{Tree}
      \begin{itemize}
        \item Both process and file are structured as tree.
        \item Process tree are usually not very deep, unlike file trees.
        \item Process hierarchy are usually short-lived (minutes or less) while directories may exist for years.
        \item Ownership and protection differs too.
      \end{itemize}
  \end{frame}

  \begin{frame}
    \frametitle{Permissions} 
    \center{\textbf{\Large{
	{\color{OliveGreen}u}
	{\color{Marroon}g}
	{\color{fuchsia}o}
        }
        :
	{\color{OliveGreen}user}
	{\color{Marroon}group}
	{\color{fuchsia}other}
      }}
      \begin{itemize}
        \item Three 3-bit fields
          \begin{itemize}
            \item r{\color{gray}ead}
            \item w{\color{gray}rite}
            \item {\color{gray}e}x{\color{gray}ecute}
          \end{itemize}
        \item {\color{OliveGreen}rwx} {\color{Marroon}rwx} {\color{fuchsia}rwx} do-what-ever-you-want-file
        \item {\color{OliveGreen}rwx} {\color{Marroon}rwx} {\color{fuchsia}r-x} web-file
        \item {\color{OliveGreen}rw-} {\color{Marroon}rwx} {\color{fuchsia}rwx} virus
        \item {\color{OliveGreen}r-x} {\color{Marroon}---} {\color{fuchsia}---} personal-backup.tgz
      \end{itemize}
    \begin{figure}
      \centering
      \begin{tabular}{l|c|c}
        Right  & File  & Directory \\ \hline
        r & can read   & can list files \\ \hline
        w & can write  & can add/delete files \\ \hline
        x & can execute& can go through \\ \hline
      \end{tabular}
      \caption{Permissions meaning}
      \label{fig:tab-right}
    \end{figure}

  \end{frame}
