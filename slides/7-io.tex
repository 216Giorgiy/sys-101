\section{Input/Ouput}

\begin{frame}
  \frametitle{I/O}
  \begin{block}{OS objectives}
    \begin{itemize}
      \item Issue commands to the devices
      \item Catch interrupts
      \item Handle errors
      \item Provide an easy-to-use interface
      \item Provide common interface for all devices
    \end{itemize}
  \end{block}
\end{frame}

\begin{frame}
  \frametitle{I/O Devices}
  \begin{itemize}
    \item block devices
    \begin{itemize}
      \item stores information in fixed-size blocks.
      \item each one with its own address.
      \item Transfers are in units of entire consecutive blocks\footnote{ranging from 512 B to 32kB}.
      \item Each block can be read or write\footnote{https://sysdig.com/50-shades-of-system-calls/} independently.
      \item Examples: USB sticks, CD-ROM, hard drives.
    \end{itemize}
    \item character devices
    \begin{itemize}
      \item delivers/accepts a stream of characters.
      \item is not addressable.
      \item cannot seek.
      \item Examples: printers, mice, network interfaces.
    \end{itemize}
  \end{itemize}
  All devices do not fit in this two-part categories. Clocks (that cause interrupts at intervals) or memory-mapped screens do not fit in.
\end{frame}

\begin{frame}
  \frametitle{I/O data rates}
    \begin{center}
      \begin{table}
        \begin{tabular}{ r | l }
          \hline
          Device      & Data rate \\ \hline \hline
          Keyboard    & 10 bytes/s \\ \hline
          52x CD-ROM  & 7.8 MB/s \\ \hline
          USB 2.0     & 60 MB/s \\ \hline
          Gigabit Eth & 125 MB/s \\ \hline
          SATA disk drive & 300 MB/s \\ \hline
          PCI bus     & 528 MB/s \\
          \hline
        \end{tabular}
        \caption{Devices and their data rate}
      \end{table}
    \end{center}
\end{frame}

\begin{frame}
  \frametitle{Memory-Mapped I/O}
  Each device has an electronic component called the device controller.

  Each controller has a few registers, used to communicate with the OS through the CPU.
\end{frame}

\begin{frame}
\frametitle{Memory-Mapped I/O}
  \begin{block}{Two means of communication}
    \begin{itemize}
      \item Each control register is assigned an I/O port number.
    \begin{itemize}
      \item overhead to read/write control registers.
      \item device driver must contain ASM.
      \item read operations require more instructions, slowing the responsiveness.
    \end{itemize}
      \item Each control register is assigned a unique memory -- this is called memory mapped I/O.
      \begin{itemize}
        \item overhead to read/write control registers.
        \item device driver can be written in C.
        \item protection is easy (the OS just don't allocate that space to user programs).
        \item Each device can (and should) have its own memory page.
        \item no caching possible.
      \end{itemize}
    \end{itemize}
  \end{block}
\end{frame}

\begin{frame}
  \frametitle{DMA\footnote{Direct Memory Access}}
  Having the CPU request one byte at a time for each device is wasteful.\\
  A DMA has access to the system bus while being CPU independent.
  \begin{block}{Without DMA}
    \begin{enumerate}
      \item the disk controller reads the block bit by bit till the whole block is in the controller's internal buffer.
      \item the checksum is compute.
      \item the controller causes an interrupt.
      \item the OS can then read the buffer word by word and store it in the memory.
    \end{enumerate}
  \end{block}
\end{frame}

\begin{frame}
  \frametitle{DMA}
  \begin{block}{With DMA}
    \begin{enumerate}
      \item The CPU programs the DMA controller through the DMA registers.
      \item The disk copies the data into its buffer
      \item The DMA controller initiates the transfer with a \emph{read} request.
      \item When the write (from buffer into memory) is complete, the disk controller sends a signal to the DMA controller.
      \item The DMA controller, once all the \emph{read}-operations are done, interrupts the CPU.
    \end{enumerate}
  \end{block}
\end{frame}

\begin{frame}
  \frametitle{DMA}
  \begin{block}{DMA differences}
    \begin{itemize}
      \item Some DMA can handle several transfers at once, each controllers having one channel for each sets of internal registers.
      \item Some DMA can operate in word-at-a-time mode and block mode.
      \item In block mode the DMA can tell the device to acquire the bus and perform a serie of transfers, then release the bus. This is called "burst mode".
    \end{itemize}
  \end{block}
  If the CPU needs the bus then it has to wait. This mechanism is called "cycle stealing".
\end{frame}

\begin{frame}
  \frametitle{DMA}
  \begin{block}{Disk read. Why this internal buffer?}
    \begin{itemize}
      \item To compute checksum.
      \item Without, when transfers start, the disk would have to write into the memory. What would happen if the data bus is busy? Words would be lost.
    \end{itemize}
  \end{block}
  Not all computer have DMA. With fast CPU, it is cheaper to deal with transfers on software level and also faster than wait for a slow DMA.
\end{frame}

\begin{frame}
  \frametitle{Interrupts}
  \begin{enumerate}
    \item A device causes an interrupt by asserting a signal on its assigned bus line.
    \item The interrupt controller detects this signal and decides what to do:
    \begin{itemize}
      \item No other interrupts are pending: the interrupt is processed immediately.
      \item Another interrupt is in process or higher-priority one has be requested: the interrupt is ignored for the moment.
    \end{itemize}
      \item The interrupt controller write a number on the address lines to specify the device causing the interrupts and sends a signal to the CPU.
  \end{enumerate}
\end{frame}

\begin{frame}
  \frametitle{Interrupts}
  \begin{enumerate}\setcounter{enumi}{3}
    \item The CPU stops what it was doing, read the number used as an index into a table called the interrupt vector to fetch a new program counter.
    \item The program counter points to the start of the interrupt service procedure.
    \item From this point, traps and interrupts use the same mechanism and usually share the same interrupt vector.
    \item After the interrupt service procedure starts, the interruption is acknowledged so the controller can fire another interrupt.
  \end{enumerate}
\end{frame}

\begin{frame}
  \frametitle{Precise Interrupt}
  Four properties characterize a precise interrupt:
  \begin{enumerate}
    \item The program counter is saved in a known place.
    \item All instructions before the one pointed to by the PC have been fully executed.
    \item No instruction beyond the PC has been executed.
    \item The execution state of the instruction pointed to by the PC is known.
  \end{enumerate}
  If an interrupt does not meet these properties, it is called an imprecise interrupt.
\end{frame}

\begin{frame}
  \frametitle{I/O Software}
  \begin{block}{Key concepts}
    \begin{itemize}
      \item Device independence: a program that reads a file should not have a different procedure to read a file from the disk, or a USB stick, or the keyboard.
      \item Uniform naming: file name should be a string and should not depend on the device.
      \item Error handling: errors should be handled as close as the hardware as possible. Why? % Faster, and more chances to correct the error
    \end{itemize}
  \end{block}
\end{frame}

\begin{frame}
  \frametitle{I/O Software}
  \begin{block}{Key concepts}
    \begin{itemize}
      \item Synchronous blocking: the OS should make I/O operations look like blocking as it is easier to program.
      \item Buffering: network packet or audio streams, just like many other, need to be buffered. The OS should take care of that.
      \item Sharable/dedicated devices: while some I/O devices can be used by many users at the same time (NIC, HDD), some cannot (printer). The OS must handle the stemming issues.
    \end{itemize}
  \end{block}
\end{frame}

\begin{frame}
  \frametitle{I/O Software}
  \begin{block}{Performing an I/O}
    \begin{itemize}
      \item Programmed I/O: the CPU does all the work so it's simple, but the polling, or busy waiting, makes it is costly.
      \item Interrupt-driven I/O: the CPU starts an I/O transfer for a character or word and goes off to do something else until an interrupt arrives signaling completion of the I/O.
      \item I/O using DMA: the DMA feed the characters to the device to let the CPU free. Interrupt number is down from one per character to one per buffer.
    \end{itemize}
  \end{block}
\end{frame}
