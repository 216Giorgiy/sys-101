\section{Input/Ouput}

\begin{frame}
  \frametitle{I/O}
  \begin{block}{OS objectives}
    \begin{itemize}
      \item Issue commands to the devices
      \item Catch interrupts
      \item Handle errors
      \item Provide an easy-to-use interface
      \item Provide common interface for all devices
    \end{itemize}
  \end{block}
\end{frame}

\begin{frame}
  \frametitle{I/O Devices}
  \begin{itemize}
    \item block devices
    \begin{itemize}
      \item stores information in fixed-size blocks.
      \item each one with its own address.
      \item Transfers are in units of entire consecutive blocks\footnote{ranging from 512 B to 32kB}.
      \item Each block can be read or write independently.
      \item Examples: USB sticks, CD-ROM, hard drives.
    \end{itemize}
    \item character devices
    \begin{itemize}
      \item delivers/accepts a stream of characters.
      \item is not addressable.
      \item cannot seek.
      \item Examples: printers, mice, network interfaces.
    \end{itemize}
  \end{itemize}
  All devices do not fit in this two-part categories. Clocks (that cause interrupts at intervals) or memory-mapped screens do not fit in.
\end{frame}

\begin{frame}
  \frametitle{I/O data rates}
    \begin{center}
      \begin{table}
        \begin{tabular}{ r | l }
          \hline
          Device      & Data rate \\ \hline \hline
          Keyboard    & 10 bytes/s \\ \hline
          52x CD-ROM  & 7.8 MB/s \\ \hline
          USB 2.0     & 60 MB/s \\ \hline
          Gigabit Eth & 125 MB/s \\ \hline
          SATA disk drive & 300 MB/s \\ \hline
          PCI bus     & 528 MB/s \\
          \hline
        \end{tabular}
        \caption{Devices and their data rate}
      \end{table}
    \end{center}
\end{frame}

\begin{frame}
  \frametitle{Memory-Mapped I/O}
  Each device has an electronic component called the device controller.

  Each controller has a few registers, used to communicate with the OS through the CPU.
\end{frame}

\begin{frame}
\frametitle{Memory-Mapped I/O}
  \begin{block}{Two means of communication}
    \begin{itemize}
      \item Each control register is assigned an I/O port number.
    \begin{itemize}
      \item overhead to read/write control registers.
      \item device driver must contain ASM.
      \item read operations require more instructions, slowing the responsiveness.
    \end{itemize}
      \item Each control register is assigned a unique memory -- this is called memory mapped I/O.
      \begin{itemize}
        \item overhead to read/write control registers.
        \item device driver can be written in C.
        \item protection is easy (the OS just don't allocate that space to user programs).
        \item Each device can (and should) have its own memory page.
        \item no caching possible.
      \end{itemize}
    \end{itemize}
  \end{block}
\end{frame}

\begin{frame}
  \frametitle{DMA\footnote{Direct Memory Access}}
  Having the CPU request one byte at a time for each device is wasteful.\\
  A DMA has access to the system bus while being CPU independent.
  \begin{block}{Without DMA}
    \begin{enumerate}
      \item the disk controller reads the block bit by bit till the whole block is in the controller's internal buffer.
      \item the checksum is compute.
      \item the controller causes an interrupt.
      \item the OS can then read the buffer word by word and store it in the memory.
    \end{enumerate}
  \end{block}
\end{frame}

\begin{frame}
  \frametitle{DMA}
  \begin{block}{With DMA}
    \begin{enumerate}
      \item the CPU programs the DMA controller through the DMA registers.
      \item When valid data are in the disk controller's buffer the DMA controller ..
      \item .. initiates the transfer with a \emph{read} request.
      \item When the write (from buffer into memory) is complete, the disk controller sends a signal to the DMA controller.
      \item The DMA controller, once all the \emph{read}-operations are done, interrupts the CPU.
    \end{enumerate}
  \end{block}
\end{frame}

\begin{frame}
  \frametitle{DMA}
  \begin{block}{DMA differences}
    \begin{itemize}
      \item Some DMA can handle several transfers at once, each controllers having one channel for each sets of internal registers.
      \item Some DMA can operate in word-at-a-time mode and block mode.
      \item In block mode the DMA can tell the device to acquire the bus and perform a series of transfers, then release the bus. This is called "burst mode".
    \end{itemize}
  \end{block}
  If the CPU needs the bus it has to wait. This mechanism is called "cycle stealing".
\end{frame}

\begin{frame}
  \frametitle{DMA}
  \begin{block}{Disk read. Why this internal buffer?}
    \begin{itemize}
      \item To compute checksum.
      \item Without, when transfers start, the disk would have to write into the memory. What would happen if the data bus is busy? Words would be lost.
    \end{itemize}
  \end{block}
  Not all computer have DMA. With fast CPU, it is cheaper to deal with transfers on software level and also faster than wait for a slow DMA.
\end{frame}

\begin{frame}
  \frametitle{Interrupts}
  \begin{itemize}
    \item Devices
    \item Without, when transfers start, the disk would have to write into the memory. What would happen if the data bus is busy? Words would be lost.
  \end{itemize}
\end{frame}
