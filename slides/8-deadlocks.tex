\section{Deadlocks}

\begin{frame}
  \frametitle{Definition by example}
  Two processes, \emph{P0} and \emph{P1}, needs to request to two resources, \emph{R0} and \emph{R1}. \emph{P0} and \emph{P1} and independent and differently programmed.
  \begin{enumerate}
    \item \emph{P1} requests, and obtains, access to \emph{R0}.
    \item \emph{P0} requests, and obtains, access to \emph{R1}.
    \item Reads/Write-operations are performed on \emph{R0} by \emph{P1} and \emph{R1} by \emph{P0}.
    \item Now \emph{P1} needs access to \emph{R1}, owned by \emph{P0}, while \emph{P0} needs access to \emph{R0}, owned by \emph{P1}.
    \item Both processes will wait the other to release the needed resource.
  \end{enumerate}
  \begin{center}
    That's a dead\textbf{lock}.
  \end{center}
\end{frame}

\begin{frame}
  \frametitle{Former definition}
  \begin{center}
    A set of processes is deadlocked if each process in the set is waiting for an event that only another process in the set can cause.
  \end{center}
\end{frame}

\begin{frame}
  \frametitle{Resources}
  \begin{block}{Resources types}
    Any object\footnote{either hardware or software} that must be acquired, used and released.
    \begin{itemize}
      \item Preemptable resource: resource that can be taken away from the process without bad effect (e.g., disk, memory).
      \item Nonpreemptable resource: resource that, if taken away, will fail the processing of a task.
    \end{itemize}
  \end{block}
  \begin{block}{Steps to use a resource}
    \begin{enumerate}
      \item Request the resource.
      \item Use the resource. (optional)
      \item Release the resource.
    \end{enumerate}
  \end{block}
\end{frame}

\begin{frame}
  \frametitle{Deadlock}
  \begin{block}{Deadlock conditions, Coffman et al. (1971)}
    \begin{enumerate}
      \item Mutual exclusion condition. Each resource is either currently assigned to exactly one process or is available.
      \item Hold and wait condition. Processes currently holding resources that were granted earlier can request new resources.
      \item No preemption condition. Resources previously granted cannot be forcibly taken away from a process. They must be explicitly released by the process holding them.
      \item Circular wait condition. There must be a circular chain of two or more processes, each of which is waiting for a resource held by the next member of the chain.
    \end{enumerate}
  \end{block}
\end{frame}

\begin{frame}
  \frametitle{Deadlock modelisation}
  \begin{block}{Schematic Representation}
    \begin{itemize}
      \item \emph{P0} \underline{requests} the resource \emph{R0}: (\emph{P0}) $\longrightarrow$ [\emph{R0}]
      \item \emph{R0} \underline{is owned} by \emph{P0}: [\emph{R0}] $\longrightarrow$ (\emph{P0})
    \end{itemize}
  \end{block}
\end{frame}

\begin{frame}
  \frametitle{Deadlock}
  \begin{enumerate}
    \item \emph{A} owns R and requests S.
    \item \emph{B} owns nothing but requests T.
    \item \emph{C} owns nothing but requests S.
    \item \emph{D} owns U and requests S and T.
    \item \emph{E} owns T and requests V.
    \item \emph{F} owns W and requests S.
    \item \emph{G} owns V and requests U.
  \end{enumerate}
  Is there a deadlock?
\end{frame}

\begin{frame}
  \frametitle{Deadlock Prevention}
  \begin{block}{Mutual Exclusion}
    Assign a resource when that is absolutely necessary. Make sure that as few processes as possible claim the resource.
  \end{block}
  \begin{block}{Hold and Wait}
    All resources are requested before the program starts. Problems stemming from this:
    \begin{enumerate}
      \item a program may not know the needed set of resources,
      \item resources will not be optimally used.
    \end{enumerate}
    Another way to break this is: a process requesting a resource first temporarily releases all the resources it currently holds, then tries to get everything it needs all at once.
  \end{block}
\end{frame}

\begin{frame}
  \frametitle{Deadlock Prevention}
  \begin{block}{No preemption}
    Virtualizing a resource can prevent deadlock. For instance a printer will not be assigned to a process but rather to its daemon, and the daemon will be assigned some disk space where processes write to print. Nevertheless, not all resources can be virtualized.
  \end{block}
  \begin{block}{Circular wait}
    If a process can own at most one single resource this condition is eliminated. But what if a process need to print a huge file sent through a serial communication? It costs too much.\\
    A solution to avoid circular wait is to order resources. If \emph{P0} and \emph{P1} must request \emph{R0} and \emph{R1} in this order then no deadlock would had occurred.
  \end{block}
\end{frame}

\begin{frame}
  \frametitle{Deadlock Prevention}
  \begin{block}{Two-Phase Locking}
    \begin{enumerate}
      \item the process tries to lock all the resources it needs, one at a time.
        \begin{itemize}
          \item On succeed: goto 2
          \item If one lock fails: release all locks, goto 1.
        \end{itemize}
      \item The process performs its updates and releases the locks. No real work is done in the first phase.
    \end{enumerate}
  \end{block}
\end{frame}

\begin{frame}
  \frametitle{Deadlock Issue}
  \begin{block}{Communication}
    Deadlocks appears also in communication system. A node S requests a resource to node R and blocks, waits, for the response. What if the request is lost?\\
    This is a communication deadlock. They cannot be prevented by ordering resources or avoided by scheduling. Timeout can prevent them.
  \end{block}
  \begin{block}{Starvation}
    When several process requests the same resource, processes may starve if new processes, arriving endlessly, always get the resource before them.\\
    Starvation can be avoided by using a first-come, first-served, resource allocation policy.
  \end{block}
\end{frame}

\begin{frame}
    \frametitle{To Read}
    \color{blue}\href{http://www.greenteapress.com/semaphores/downey08semaphores.pdf}{The Little Book of Semaphores}

\end{frame}
