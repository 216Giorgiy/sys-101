\section{System Calls}
\begin{frame}
  \frametitle{System calls}
  \begin{block}{System operations}
    \begin{itemize}
      \item Read file, create directory or a process, modify permissions ..
      \item .. results in a system call.
    \end{itemize}
  \end{block}
  \begin{block}{System Calls}
      Interface between user space and kernel space.
  \end{block}
\end{frame}

\begin{frame}
  \frametitle{System calls}
  \begin{block}{System calls procedure}
    \begin{enumerate}
      \item Switch from the program, within the user space, into the kernel space by executing a trap instruction.
      \item The program starts the execution at a \textbf{fixed} address in the kernel space.
      \item The kernel code then examine the system call number and execute the matching system call handler.
      \item The system call handler execute its code and then returns the control to the program at the instruction following the trap instruction.
        \begin{itemize}
          \item unless the system call blocks (i.e., waiting for an input on the keyboard).
        \end{itemize}
      \item The program is put back into user space and continues.
    \end{enumerate}
  \end{block}
\end{frame}

\begin{frame}
  \frametitle{Main system calls: Process management}
  \begin{figure}
    \centering
    \begin{tabular}{l|c}
      Call                  & Description \\ \hline
      {\scriptsize pid = fork()}          & Create a child process \\ & (identical to the parent) \\ \hline
      {\scriptsize pid = waitpid(pid, \&stat, opt)} &  Waif for a child to terminate \\ \hline
      {\scriptsize s = execve(name, argv, envp)} & Execute a program \\ \hline
      {\scriptsize exit(status)} & Terminate a process and return status \\ \hline
    \end{tabular}
    \caption{Process management}
    \label{fig:sys-call_process}
  \end{figure}
\end{frame}

\begin{frame}
  \frametitle{Main system calls: File management}
  \begin{figure}
    \centering
    \begin{tabular}{l|c}
      Call                  & Description \\ \hline
      {\scriptsize fd = open(file, flags)}          & Open (or create) a file \\ \hline
      {\scriptsize close(fd)} &  Close a file \\ \hline
      {\scriptsize n = read(fd, buff, nbytes)} & Read data from a file into a buffer \\ \hline
      {\scriptsize n = write(fd, buff, nbytes)} & Write data from a buffer into a file \\ \hline
      {\scriptsize p = lseek(fd, offset, whence)} & Reposition the pointer within the file \\ \hline
      {\scriptsize s = stat(name, buff)} & Get file status \\ \hline
    \end{tabular}
    \caption{File management}
    \label{fig:sys-call_file}
  \end{figure}
\end{frame}

\begin{frame}
  \frametitle{Main system calls: Directory and file system management}
  \begin{figure}
    \centering
    \begin{tabular}{l|c}
      Call                  & Description \\ \hline
      {\scriptsize s = mkdir(name, mode)}          & Create a directory \\ \hline
      {\scriptsize s = rmdir(name)}          & Delete a directory \\ \hline
      {\scriptsize s = link(oldpath, newpath)}          & Make a new name for a file \\ \hline
      {\scriptsize s = unlink(path)}          & Delete a name and possibly the file it refers to \\ \hline
      {\scriptsize s = mount(s, t, fst, f, d)}          & Mount a filesystem \\ \hline
      {\scriptsize s = umount(target)}          & Unmount a filesystem \\ \hline
    \end{tabular}
    \caption{Directory and file system management}
    \label{fig:sys-call_dir}
  \end{figure}
\end{frame}


\begin{frame}
  \frametitle{Main system calls: Miscellaneous management}
  \begin{figure}
    \centering
    \begin{tabular}{l|c}
      Call                  & Description \\ \hline
      {\scriptsize s = chdir(dirname)}          & Change the working directory \\ \hline
      {\scriptsize s = chmod(name, mode)}       & Change a file's protection bits \\ \hline
      {\scriptsize s = kill(pid, signal)}       & Send a signal to a process \\ \hline
      {\scriptsize seconds = time(\&seconds)}    & Get the elapsed time since 01/01/70 \\ \hline
    \end{tabular}
    \caption{Miscellaneous}
    \label{fig:sys-call_misc}
  \end{figure}
\end{frame}

\begin{frame}
  \frametitle{Q/A}
  \begin{itemize}
    \item What is a trap instruction, its difference with an interuption? % there is not sync between a program and an interruption, whereas a trap instruction is performed by a program itself.
    \item What is the purpose of a system call in an operating system? % sys-calls allow a user process to access kernel function.
    \item Reading a file is a blocking operation. Should writing a file be a blocking operation too? % Depends: if the program overrides the buffer right after the call the written data may be corrupted. If the kernel copies the buffer before returning into the program it does not have to block. Also, what about returning an error if not blocking?
    \item If {\scriptsize open()} and {\scriptsize close()} would be removed from the system calls, what changes would be implied?
    \item What are the buffered data after these instructions:
    \begin{itemize}
        \item lseek(df, 3, SEEK\_SET);
        \item read(df, buffer, 4);
   \end{itemize}
   if the pointed file df content is: 3,5,8,9,7,5,6,2,1,4 ?
     % 9,7,5,6
  \end{itemize}
\end{frame}
