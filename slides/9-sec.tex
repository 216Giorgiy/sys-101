\section{Security}

  \begin{frame}
      \frametitle{Security}
      \begin{block}{OS Security Goals vs. Threat}
        \begin{itemize}
          \item Data confidentiality: unauthorized users should not be able to access any file.
          \item Data integrity: unauthorized users should not be able to modify any data.
          \item System availability: nobody should be able to bring the system in an unusable state (i.e., DDOS, soft-bomb).
          \item System integrity: unauthorized users should not be able to access a system.
          \item Privacy: so much to say, but out of the scope.
          \item Accidental loss: valuable data can be lost by accident.
        \end{itemize}
      \end{block}
  \end{frame}

    \begin{frame}
        \frametitle{Security}
        \begin{block}{Intruders}
          Two different ways to act:
          \begin{itemize}
            \item Passive: just read data they are not authorized to.
            \item Active: add, remove, modify data.
          \end{itemize}
          Can be categorize as:
          \begin{itemize}
            \item Casual nontechnical.
            \item Snooping by insiders. Highly skilled taking security as a challenge.
            \item Determined attempts to make money.
            \item Commercial or military espionage.
          \end{itemize}
        \end{block}
    \end{frame}

    \begin{frame}
        \frametitle{Protection mechanisms}
        \begin{block}{Protection Domains}
          How to prohibit processes from accessing objects that they are not authorized to access?
          \begin{itemize}
            \item A domain is a set of (object, rights\footnote{subset of the operations that can be performed on it}) pairs.
            \item A domain may correspond to a single user or a group.
            \item Every process runs in a protection domain. The objects within this same domain are accessible. UNIX domains are defined by the UID and the GID.
            \item Each process has two halves: user part and kernel part. Each part has different set of accessible objects.
          \end{itemize}
        \end{block}
        Domains can be stored in matrix, but it is usually space wasting as most domains do not have any right on most objects.
    \end{frame}

    \begin{frame}
        \frametitle{Protection mechanisms}
        \begin{block}{Access Control List}
          This matrix can be stored as a list, either by row\footnote{then name \textbf{capabilities}.} or column, with only non-empty elements.
        \end{block}
        \begin{block}{Trusted System}
          \begin{itemize}
            \item Microsoft has one\footnote{Frandrich \emph{et al.}, 2006} but do not sell it.
            \item Why don't we use one?
            \begin{itemize}
              \item Because features (ASCII email, WWW).
              \item Design a system as simple as possible is required.
            \end{itemize}
          \end{itemize}
        \end{block}
    \end{frame}
