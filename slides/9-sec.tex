\section{Security}

  \begin{frame}
      \frametitle{Security}
      \begin{block}{OS Security Goals vs. Threat}
        \begin{itemize}
          \item Data confidentiality: unauthorized users should not be able to access any file.
          \item Data integrity: unauthorized users should not be able to modify any data.
          \item System availability: nobody should be able to bring the system in an unusable state (i.e., DDOS, soft-bomb).
          \item System integrity: unauthorized users should not be able to access a system.
          \item Privacy: so much to say, but out of the scope.
          \item Accidental loss: valuable data can be lost by accident.
        \end{itemize}
      \end{block}
  \end{frame}

    \begin{frame}
        \frametitle{Security}
        \begin{block}{Intruders}
          Two different ways to act:
          \begin{itemize}
            \item Passive: just read data they are not authorized to.
            \item Active: add, remove, modify data.
          \end{itemize}
          Can be categorize as:
          \begin{itemize}
            \item Casual nontechnical.
            \item Snooping by insiders. Highly skilled taking security as a challenge.
            \item Determined attempts to make money.
            \item Commercial or military espionage.
          \end{itemize}
        \end{block}
    \end{frame}

    \begin{frame}
        \frametitle{Protection mechanisms}
        \begin{block}{Protection Domains}
          How to prohibit processes from accessing objects that they are not authorized to access?
          \begin{itemize}
            \item A domain is a set of (object, rights\footnote{subset of the operations that can be performed on it}) pairs.
            \item A domain may correspond to a single user or a group.
          \end{itemize}
        \end{block}
    \end{frame}
